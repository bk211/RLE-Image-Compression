\documentclass[12pt, letterpaper]{article}
\usepackage[utf8]{inputenc}
\usepackage{graphicx}

\title{Rapport de projet : Compression d'image par la méthode Run-Length-Encoding}
\author{Chaolei Cai
\\
    \multicolumn{1}{
        p{.7\textwidth}}{\centering\emph{Université Paris Vincennes St-Denis\\
  UFR mathématiques, informatique, technologies sciences de l'information\\}
  L3 Informatique}
}
\date{\today}
\begin{document}


\begin{titlepage}
    \maketitle
\end{titlepage}

\tableofcontents

\section{Présentation du projet}
Ce document est mon rapport pour le projet 11 du cours d'algorithmique avancé enseigné par M.Bourdin.\\
Le sujet de mon projet est de mettre en place l'algorithme de Run-Length-Encoding (RLE).\\
Il est demandé de satisfaire ces points suivants.
\begin{itemize}
    \item Tester le RLE à partir des trois plans R, puis G, puis B. 
    \item Tester le RLE en ayant transformé l'image en mode HSV et en séparant les champs de H, de S et de V. 
    \item Écrire dans chaque cas la fonction qui code en mode RLE, la fonction qui fait la sauvegarde et la fonction qui permet d'afficher une image ainsi sauvegardée. 
    \item Essayer votre méthode sur un bon nombre d'images et comparer les résultats obtenus, en particulier par rapport à des méthodes concurrentes (LZW...).
\end{itemize}

%\begin{figure}
%    \includegraphics[width=\linewidth]{img/L1.png}
%    \caption{Welcome page}
%    \label{fig:L1}
%\end{figure}


\subsection{Profile}
%\includegraphics[width=\linewidth]{img/L6.png}
 

\end{document}
